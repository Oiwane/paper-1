\chapter{はじめに}\label{cha:Introduction}
近年ソフトウェア開発の効率化が求められており、設計段階や製品を試作する前に、製品の機能や性能を検証したいというニーズが高まっている\cite{modelicaモデルベース本}。
このニーズに応えるために、モデルベースシステム開発手法がある\cite{modelicaモデルベース本}。モデルベースシステム開発手法とは、製品の設計を元にシミュレーションツールを用いて、
シミュレーションを行いながら、設計品質の向上を図る開発手法である\cite{ipa_2016}。設計の品質が向上することによって、不具合が減り、不具合が減ることによって、手戻りを削減できるため、生産性の向上が期待できる\cite{ipa_2016}。

モデルベースシステム開発手法は、 組込みシステムの開発において特に重要である\cite{ipa_useful_modelbase_dev}。製品開発を行う際に、モデルベースシステム開発手法を適用することで、
実際に製品を試作することなく、製品の様々な性能を事前に確認できるため、開発コストを削減できる\cite{modelicaモデルベース本}。また、実際に試作品を作る回数を減らすことができるため、
製品開発の納期の短縮が期待できる。

モデルベースシステム開発手法を用いた組込みシステムの開発に、OpenModelica\cite{open_modelica}が使われる。
OpenModelicaは、Modelica\cite{modelicaモデルベース本}コードのモデリング、シミュレーション、デバッグのための機能などを持つオープンソースプラットフォームである。
OpenModelicaが出力するシミュレーションの結果は、グラフや数値であり、csvファイルに出力できる。しかし、この出力を用いて、性能を決定付ける特定の値を確認するためには、手間と時間がかかる。
そこで、本研究では、性能を決定付ける特定の値を確認するためにかかる時間の削減を目的として、OpenModelicaのシミュレーション結果を用いた特性表の自動生成ツールの試作を行う。
特性表とは、製品の性能をまとめた一覧表であり、複数の製品の性能比較をする際に利用される[参考文献]。特性表を用いることで、性能の特定の値を容易に確認できるため、本研究の生成対象とする。なお、本研究では、シミュレーションの対象として、ブラシ付きDCモータ[モータ使う]を対象とする。


モデルベースシステム開発手法を用いた組込みシステムの開発に、OpenModelica\cite{open_modelica}が使われる。OpenModelicaは、
Modelica\cite{modelicaモデルベース本}コードのモデリング、シミュレーション、デバッグのための機能などを持つオープンソースプラットフォームである。
OpenModelicaが出力するシミュレーションの結果は、グラフや数値であり、csvファイルに出力できる。しかし、この出力を用いて、性能を決定付ける特定の値を確認するためには、手間と時間がかかる。
そこで、本研究では、性能を決定付ける特定の値を確認するためにかかる時間の削減を目的として、OpenModelicaのシミュレーション結果を用いた特性表の自動生成ツールの試作を行う。特性表とは、製品の性能をまとめた表であり、
特性表を用いることで、特定の値を容易に確認できる。なお、本研究では、シミュレーションの対象として、ブラシ付きDCモータ\cite{モータ使う}を対象とする。

% この問題を解決する手段の1つとして、特性表を用いることが考えられる。しかし、特性表の作成は、人手により作成するため、手間と時間を必要とする。OpenModelicaのシミュレーション結果を用いて、特性表を作成するためには、OpenModelicaが出力したcsvファイルから特定の値を算出し、グラフを生成する必要がある。

本論文の構成は、以下の通りである。\\
第2章では、モータ特性表自動生成ツールを試作するために必要となる前提知識について説明する。\\
第3章では、試作したモータ特性表自動生成ツールの構成および実装について説明する。\\
第4章では、試作したモータ特性表自動生成ツールが正しく動作することを検証する。\\
第5章では、試作したモータ特性表自動生成ツールについて考察する。\\
第6章では、本論文のまとめと今後の課題を述べる。\\