\chapter{はじめに}\label{cha:Introduction}
近年、モータはエアコン・洗濯機・掃除機などの家電製品をはじめ、自動車関係、医療関係など様々な分野に
用いられており\cite{モータ使用製品}、社会に必要不可欠な存在となっている。\\

~はじめに 流れ 案1~\\
シミュレーションを行った場合、期待通りか確認する。\\
確認する際は、シミュレーション結果から目的のグラフや値を計算等して作成しなければならない。\\
今回試作したツールで、グラフや値を作成する手間を省くことで、モータ開発の効率化を図る。\\



本論文の構成は、以下の通りである。\\
第2章では、試作したモータ特性表自動生成ツールを開発するために必要となる前提知識について説明する。\\
第3章では、試作したモータ特性表自動生成ツールの機能について説明する。\\
第4章では、モータ特性表自動生成ツールの実装について説明する。\\
第5章では、試作したモータ特性表自動生成ツールの機能が正しく動作することを検証する。\\
第6章では、試作したモータ特性表自動生成ツールについて考察する。\\
第7章では、本論文のまとめと今度の課題を述べる。\\
