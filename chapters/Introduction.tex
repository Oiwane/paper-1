\chapter{はじめに}\label{cha:Introduction}

近年、デジタルゲームの開発人口が増加傾向にあり[1]、ゲーム市場の競争率が上がっている[2]ことから、デジタルゲーム開発の大規模化、複雑化が進んでいる[3]。このような背景から、
開発したゲームに対するアドホックテスト[4]を、開発チームとは別の複数人のテスターが行う場合が多い[5]。
しかし、開発チームとテスターが異なる場合、以下の2つの問題点が存在する。

\begin{itemize}
	\item テスターがバグを発見した際の、報告書や口頭による開発チームへの説明は人為的なミスが入り込む可能性がある
	\item テスターが発見したバグを開発チームが再現するために、テスターが行った入力を再現することは手間と時間がかかってしまう
\end{itemize}

本研究では、上記2つの問題点を解決することを目的として、
ゲーム開発におけるテスト支援ツールを試作する。

試作するテスト支援ツールは、前者の問題点を解決するために、テストの報告書を自動生成する。
また、後者の問題点を解決するために、入力履歴を自動で保存し、その保存した入力履歴を用いて自動で再現する。
これら2つの解決策により、ゲーム開発における、テスト効率の向上を図る。
試作するテスト支援ツールは、具体的に以下の2つの特徴を持つ。
\begin{itemize}
	\item キャラクターとオブジェクトの重なり判定を自動で行い、報告書としてファイルに出力
	\item プレイしたゲームの自動リプレイ
\end{itemize}

ここで、キャラクターとは、ゲームにおいてユーザーが操作できる画像であり、
オブジェクトとは、ユーザーが選択できるゲーム画面上の任意の大きさの四角形である。

本論文の構成は、以下の通りである。\\
第2章では、試作したテスト支援ツールを開発するために必要となる前提知識について説明する。\\
第3章では、試作したテスト支援ツールの概観と機能について説明する。\\
第4章では、テスト支援ツールの実装について説明する。\\
第5章では、試作したテスト支援ツールの機能が正しく動作することを検証する。\\
第6章では、試作したテスト支援ツールについて考察する。\\
第7章では、本論文のまとめと今度の課題を述べる。\\

