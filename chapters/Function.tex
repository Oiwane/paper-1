\chapter{機能}\label{cha:Function}

この章では、本研究で試作したモータ特性表自動生成ツールの機能について説明する。\\
モータ特性表自動生成ツールは、OpenModelicaでモータのモデルをシミュレーションした時に出力される
csvファイルを読み込み、実行することによって、モータ特性表を生成する。\\

\section{対応するモデル}\label{pre_kinou}
今回試作したモータ特性表自動生成ツールでは以下のmodelicaモデルに対応する。
\begin{itemize}
	\item モータ単体のモデル
	\item モータ単体のモデルを一つに統一したモデルを使用するモデル
\end{itemize}
以降、具体的に説明する。

\subsection{モータ単体のモデル}\label{sec:sub1}
モータ単体のモデルとは、

\subsection{モータモデルを一つにまとめたモデル}\label{sec:sub2}




% % \begin{figure}
% %   \centering
% %   \includegraphics[width=10cm]{./Image/3.4.eps}
% %   \caption{オブジェクト位置確認ウィンドウの概観}
% % \end{figure}


\section{特性表生成}\label{kenkyu_mokuteki}


