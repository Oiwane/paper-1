\chapter{適用例}\label{cha:Indication}
本章では、本研究で作成した


\section{モータ単体のモデル}

% % \begin{figure}
% % 	\begin{minipage}[t]{0.5\columnwidth}
% % 		\begin{center}
% % 			\includegraphics[clip, width=0.8\columnwidth]{./Image/5.1.eps}
% % 		\end{center}
% % 		\caption{キャラクター情報を入力する前の \protect\newline キャラクター情報入力ウィンドウ}
% % 		\label{fig:left}
% % 	\end{minipage}%
% % 	\begin{minipage}[t]{0.5\columnwidth}
% % 		\begin{center}
% % 			\includegraphics[clip, width=0.8\columnwidth]{./Image/5.2.eps}
% % 		\end{center}
% % 		\caption{キャラクター情報を入力した後の \protect\newline キャラクター情報入力ウィンドウ}
% % 		\label{fig:right}
% % 	\end{minipage}
% % \end{figure}

% % \begin{figure}
% %   \centering
% %   \includegraphics[width=10cm]{./Image/5.3.eps}
% %   \caption{入力したキャラクター名とキャラクターのファイルパスを保存しているcsvファイル}
% % \end{figure}

% キャラクター情報を入力する前のキャラクター情報入力ウィンドウと、キャラクター情報を入力した後のキャラクター情報入力ウィンドウを比較すると、キャラクター画像(ball.jpg)とキャラクター名(Player)を
% 入力できている。また、csvファイルの1行目に、キャラクター名(Player)とキャラクター画像(ball.jpg)のファイルパスが記述していることが
% わかる。よって、「キャラクター情報入力機能」が正しく動作していることを確認できる。

\section{パッケージ化されたモデル}
% テスト支援ツールにおいて、「オブジェクト増減機能」が機能することを検証するために、「オブジェクトを追加する前」と、「オブジェクトを追加した後」、「オブジェクトが減少した後」のテスト支援ツールのウィンドウを比較する。
% オブジェクトを追加する前のテスト支援ツールのウィンドウを図5.4に、オブジェクトを追加した後のテスト支援ツールのウィンドウを図5.5に、それぞれ示す。また、オブジェクトが減少した後のテスト支援ツールのウィンドウを、図5.6に示す。

% % \begin{figure}
% % 	\begin{minipage}[t]{0.5\columnwidth}
% % 		\begin{center}
% % 			\includegraphics[clip, width=0.9\columnwidth]{./Image/5.4.eps}
% % 		\end{center}
% % 		\caption{オブジェクトを追加する前の \protect\newline テスト支援ツールのウィンドウ}
% % 		\label{fig:left}
% % 	\end{minipage}%
% % 	\begin{minipage}[t]{0.5\columnwidth}
% % 		\begin{center}
% % 			\includegraphics[clip, width=0.9\columnwidth]{./Image/5.5.eps}
% % 		\end{center}
% % 		\caption{オブジェクトを追加した後の \protect\newline テスト支援ツールのウィンドウ}
% % 		\label{fig:right}
% % 	\end{minipage}
% % \end{figure}

% % \begin{figure}
% %   \centering
% %   \includegraphics[width=6cm]{./Image/5.6.eps}
% %   \caption{オブジェクトが減少した後のテスト支援ツールのウィンドウ}
% % \end{figure}

% オブジェクトを追加する前と、オブジェクトを追加した後を比較すると、オブジェクトが増加していることがわかる。
% また、オブジェクトを追加した後とオブジェクトが減少した後を比較すると、オブジェクトが減少していることがわかる。
% よって、「キャラクター情報入力機能」が正しく動作していることを確認できる。


% \section{オブジェクト情報入力機能}
% テスト支援ツールにおいて、「オブジェクト情報入力機能」が機能することを検証するために、「オブジェクト情報を入力する前」と、「オブジェクト情報を入力した後」のオブジェクト情報入力ウィンドウを比較する。
% オブジェクト情報を入力する前のオブジェクト情報入力ウィンドウを図5.7に、オブジェクト情報を入力した後のオブジェクト情報入力ウィンドウを図5.8に、それぞれ示す。

% % \begin{figure}
% % 	\begin{minipage}[t]{0.5\columnwidth}
% % 		\begin{center}
% % 			\includegraphics[clip, width=0.9\columnwidth]{./Image/5.7.eps}
% % 		\end{center}
% % 		\caption{オブジェクト情報を入力する前の \protect\newline オブジェクト情報入力ウィンドウ}
% % 		\label{fig:left}
% % 	\end{minipage}%
% % 	\begin{minipage}[t]{0.5\columnwidth}
% % 		\begin{center}
% % 			\includegraphics[clip, width=0.9\columnwidth]{./Image/5.8.eps}
% % 		\end{center}
% % 		\caption{オブジェクト情報を入力した後の \protect\newline オブジェクト情報入力ウィンドウ}
% % 		\label{fig:right}
% % 	\end{minipage}
% % \end{figure}


% オブジェクト情報を入力する前と、オブジェクト情報を入力した後のオブジェクト情報入力ウィンドウを比較すると、オブジェクト情報が入力していることがわかる。
% よって、「オブジェクト情報入力機能」が正しく動作していることを確認できる。

% \section{オブジェクト位置確認機能}
% テスト支援ツールにおいて、「オブジェクト位置確認機能」が機能することを検証するために、4.7節で説明した「ゲーム実行時のゲーム画面リアルタイム取得」
% で取得した画像に、実際のオブジェクトの位置を表示し、それとゲーム背景画像が等しいかどうかを比較する。
% 実際のオブジェクトの位置を図5.9に、ゲーム背景画像を図5.10に、それぞれ示す。
% なお、これらの図は、オブジェクトの位置がわかりやすいように、オブジェクトの位置の線を強調している。

% % \begin{figure}
% % 	\begin{minipage}[t]{0.5\columnwidth}
% % 		\begin{center}
% % 			\includegraphics[clip, width=0.9\columnwidth]{./Image/5.9.eps}
% % 		\end{center}
% % 		\caption{実際のオブジェクトの位置}
% % 		\label{fig:left}
% % 	\end{minipage}%
% % 	\begin{minipage}[t]{0.5\columnwidth}
% % 		\begin{center}
% % 			\includegraphics[clip, width=0.9\columnwidth]{./Image/5.10.eps}
% % 		\end{center}
% % 		\caption{ゲーム背景画像}
% % 		\label{fig:right}
% % 	\end{minipage}
% % \end{figure}

% 実際のオブジェクトの範囲とゲーム背景画像を比較すると、オブジェクトの位置が同じことがわかる。
% よって、「オブジェクト位置確認機能」が正しく動作していることを確認できる。

% \section{Warning判定出力機能}
% テスト支援ツールにおいて、「Warning判定出力機能」が機能することを検証する。
% オブジェクトの位置を指定した場所に、キャラクターが重なった場合、出力するファイルに、重なったオブジェクト名が表示しているかを確認する。
% 今回は、5.4節のオブジェクト位置確認機能を使い、オブジェクトの位置が設定できたことを確認する。
% オブジェクト位置確認ウィンドウを図5.11に、オブジェクトと報告者が設定しているツールの概観を図5.12に、それぞれ示す。
% 図5.11には、今回設定したオブジェクト名をオブジェクトの位置に記述している。
% 図5.11のゲーム背景の左からオブジェクトの名前を"Object1", "Object2", "Object3"とする。オブジェクトのそれぞれの座標を、表5.1に示す。


% % \begin{figure}[t]
% %   \centering
% %   \includegraphics[width=8cm]{./Image/5.11.eps}
% %   \caption{オブジェクト位置確認ウィンドウ}
% % \end{figure}

% % \begin{figure}[t]
% %   \centering
% %   \includegraphics[width=6cm]{./Image/5.12.eps}
% %   \caption{オブジェクトと報告者とキャラクターを設定してるツールの概観}
% % \end{figure}

% \begin{table}[t]
%   \centering
%   \caption{オブジェクトの座標}
%   \begin{tabular}{|l|l|l|l|l|} \hline
%     オブジェクト名 & 左上の座標x & 左上の座標y & 右下の座標x & 右下の座標y  \\ \hline \hline
% %        Object1 & 0 & 0 & 200 & 800\\ \hline
% %        Object2 & 600 & 0 & 800 & 800 \\ \hline
% % 	   Object3 & 1000 & 0 & 1400 & 800 \\ \hline
% %   \end{tabular}
% % \end{table}

% % キャラクターを操作して"Object1→Object2→Object3"の順番にオブジェクトと重ねていく。
% % この時、出力する報告書の例を、図5.13に示す。

% % % \begin{figure}[t]
% % %   \centering
% % %   \includegraphics[width=6cm]{./Image/5.13.eps}
% %   \caption{報告書の例}
% % \end{figure}

% 図5.13より、報告者、キャラクターと重なったオブジェクト名、および、重なった時の時刻を、報告書に正しく記述していることがわかる。
% よって、「Warning判定出力機能」が正しく動作することを確認できる。

% \section{入力キーとタイミングの記録機能}
% テスト支援ツールにおいて、「入力キーとタイミングの記録機能」が機能することを検証するために、
% ゲームを開始した後に、入力するキーを「左矢印キー、上矢印キー、右矢印キー、下矢印キー、Aキー、Sキー、Dキー、Wキー」の順にキーを押下して、出力するリプレイファイルを確認する。
% 出力したリプレイファイルを、図5.14に示す。

% % \begin{figure}[t]
% %   \centering
% %   \includegraphics[width=5cm]{./Image/5.14.eps}
% %   \caption{リプレイファイルの例}
% % \end{figure}

% 図5.14のリプレイファイルを確認すると、入力した通りにキーの種類と、その押下した時刻を正しく記録していることがわかる。
% よって、「入力キーとタイミングの記録機能」が正しく動作することを確認できる。

% \section{リプレイ機能}
% テスト支援ツールにおいて、「リプレイ機能」が機能することを検証するために、ゲームをスタートして自分で動かしたキャラクターと、
% リプレイ機能を使って自動で移動するキャラクターの動きを比較する。
% 今回は入力するキーを「右矢印キー、右矢印キー」とする。
% ゲームをスタートして自分で動かしたキャラクターの動きを図5.15に、リプレイ機能を使って自動で移動するキャラクターの動きを図5.16に、それぞれ示す。


% 図5.15と図5.16を比べると、キャラクターの動きは最初に指定した、「右矢印キー、右矢印キー」となっている。
% しかし、最後のゲーム画面のキャラクターの位置を確認すると、誤差が生じていることがわかる。
% この結果について、考えられる原因と解決策を、6章の考察で述べる。


% \section{エラーメッセージ出力機能}
% テスト支援ツールにおいて、「エラーメッセージ出力機能」が機能することを検証するために、以下の状態の場合に、
% エラーメッセージが表示することを確認する。
% \begin{enumerate}
%   \item ゲームスタートボタンを押下した際に、実行するゲームファイル、報告者、キャラクターを設定していない。
%   \item リプレイボタンを押下した際に実行するゲームファイルや入力したキーとタイミングが記録しているファイルを設定していない。
%   \item 5つ以上のオブジェクトを追加する。
% \end{enumerate}

% % \begin{figure}[t]
% %   \centering
% %   \includegraphics[width=15cm]{./Image/5.15.eps}
% %   \caption{自分で動かしたキャラクターの動き}
% % \end{figure}

% % \begin{figure}[t]
% %   \centering
% %   \includegraphics[width=15cm]{./Image/5.16.eps}
% %   \caption{リプレイ機能を使って自動で移動するキャラクターの動き}
% % \end{figure}

% 図5.17に1.の状態の時、図5.18に2.の状態の時、図5.19に3.の状態の時の、それぞれのエラーメッセージの表示を示す。
% % \begin{figure}
% %   \centering
% %   \includegraphics[width=17cm]{./Image/5.17.eps}
% %   \caption{1.の状態の時のエラーメッセージ}
% % \end{figure}

% % \begin{figure}
% %   \centering
% %   \includegraphics[width=13cm]{./Image/5.18.eps}
% %   \caption{2.の状態の時のエラーメッセージ}
% % \end{figure}

% % \begin{figure}
% %   \centering
% %   \includegraphics[width=13cm]{./Image/5.19.eps}
% %   \caption{3.の状態の時のエラーメッセージ}
% % \end{figure}

% 図5.17、図5.18、図5.19の初期状態のウィンドウと、エラーメッセージを表示しているウィンドウを比較すると、
% 状態に応じて、エラーメッセージを正しく表示していることが確認できる。よって、「エラーメッセージ出力機能」が正しく動作していることを確認できる。

% %\input{CommonTexs/Items_Area}
% %以降、それぞれの処理部について説明する。



