\chapter{おわりに}\label{cha:Conclusion}
本論文では、性能を決定付ける特定の値を確認するためにかかる時間の削減を目的として、モータ特性表自動生成ツールを試作した。
なお、本研究では、シミュレーションの対象として、ブラシ付きDCモータを対象とする。

モータ特性表自動生成ツールは、csvファイル解析部、特性表の要素算出部、モータ特性表生成部の3つの処理部で構成している。
csvファイル解析部では、csvファイルを読み込み、モータ特性表を生成するために必要なデータを抽出する。特性表の要素算出部では、抽出したデータからモータ特性表の各要素を算出する。
モータ特性表生成部では、算出したデータを基に特性表と、4つの特性グラフを生成し、これらを1つのPDFファイルにまとめ、モータ特性表として出力する。

適用例として、「ブラシ付きDCモータのModelicaモデル」と「ブラシ付きDCモータのModelicaモデルをサブシステムとするモデル」のシミュレーション結果ファイルを適用した結果、
2つのモデルから正しくモータ特性表を生成することを確認した。
よって、モータ特性表自動生成ツールは、「ブラシ付きDCモータのModelicaモデル」と「ブラシ付きDCモータのModelicaモデルをサブシステムとするモデル」に対応していることが確認できる。

考察の評価において、モータ特性表自動生成ツールの有用性を示すことができた。具体的には、モータ特性表自動生成ツールを用いることにより、特定の値を確認するためにかかる時間を削減できるかどうかを検証した。
検証にはXとYの2種類のケースを用意し、被験者4名を2グループに分けて、モータ特性表自動生成ツールを用いる場合と、用いない場合で実験を行った。

この実験結果により、ケースXについてはモータ特性表自動生成ツールを使用した場合では、使用しなかった場合に比べて被験者の回答時間を91.8\%削減できた。
ケースYについては、モータ特性表自動生成ツールを使用した場合では、使用しなかった場合に比べて被験者の回答時間を92.1\%削減できた。
この結果により、特定の値を確認するためにかかる時間を削減できたと言える。また、双方のケースでモータ特性表自動生成ツールを用いた場合、モータ特性表自動生成ツールを用いない場合に比べ、問題の正答率が上昇した。
% 正答率が上がった理由として、モータ特性表自動生成ツールを用いる
% 人手によるミスを削減できたことが考えられる。具体的には、モータ特性表自動生成ツールを用いない場合、手動で表計算ソフトから特定の値を抽出し、計算を行う必要があるため、
% この過程においてミスが発生する可能性が高まったことが考えられる。
% 一方、モータ特性表自動生成ツールを用いた場合、ツールが自動で特性表を生成し、提示することにより、人手によるミスが発生する可能性を削減できたことが考えられる。

以下に、今後の課題を示す。




