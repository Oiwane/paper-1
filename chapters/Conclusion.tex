\chapter{おわりに}\label{cha:Conclusion}
本論文では、性能を決定付ける特定の値を確認するためにかかる時間の削減を目的として、OpenModelicaのシミュレーション結果を用いたモータ特性表自動生成ツールを試作した。
なお、本研究では、シミュレーションの対象として、ブラシ付きDCモータを対象とする。

モータ特性表自動生成ツールは、csvファイル解析部、特性表の要素算出部、モータ特性表生成部の3つの処理部で構成している。
csvファイル解析部では、csvファイルを読み込み、モータ特性表を生成するために必要なデータを抽出する。特性表の要素算出部では、抽出したデータからモータ特性表の各要素を算出する。
モータ特性表生成部では、算出した各要素を基に、特性表と4つの特性グラフを生成し、これらを1つのPDFファイルにまとめ、モータ特性表として出力する。

適用例として、ブラシ付きDCモータのModelicaモデルのシミュレーション結果であるcsvファイルを用いる。このcsvファイルをモータ特性表自動生成ツールに適用した結果、
モータ特性表を正しく生成することを確認した。
% よって、モータ特性表自動生成ツールは、「ブラシ付きDCモータのModelicaモデル」と「ブラシ付きDCモータのModelicaモデルをサブシステムとするモデル」に対応していることが確認できる。

考察において、モータ特性表自動生成ツールの有用性を示すことができた。具体的には、モータ特性表自動生成ツールを用いることにより、特定の値を確認するためにかかる時間を削減できることを検証した。
検証にはケースXとケースYの2種類のケースを用意し、被験者4名を2グループに分けて、モータ特性表自動生成ツールを用いる場合と、用いない場合で実験を行った。

この実験結果により、モータ特性表自動生成ツールを使用した場合では、使用しなかった場合に比べて被験者の解答時間を92.0\%削減できた。
% ケースYについては、モータ特性表自動生成ツールを使用した場合では、使用しなかった場合に比べて被験者の回答時間を92.1\%削減できた。
この結果により、特定の値を確認するためにかかる時間を削減できたと言える。また、双方のケースでモータ特性表自動生成ツールを用いた場合は、モータ特性表自動生成ツールを用いない場合に比べて、正答率が高かった。
% 正答率が上がった理由として、モータ特性表自動生成ツールを用いる
% 人手によるミスを削減できたことが考えられる。具体的には、モータ特性表自動生成ツールを用いない場合、手動で表計算ソフトから特定の値を抽出し、計算を行う必要があるため、
% この過程においてミスが発生する可能性が高まったことが考えられる。
% 一方、モータ特性表自動生成ツールを用いた場合、ツールが自動で特性表を生成し、提示することにより、人手によるミスが発生する可能性を削減できたことが考えられる。

以上の結果により、モータ特性表自動生成ツールの有用性があることを示せた。また、モータ特性表自動生成ツールを用いることで、特定の値を確認する際に生じるミスを削減できることが確認できた。

以下に、今後の課題を示す。

\begin{itemize}
    \item 対象とするモータのモデルの拡大\\
    本論文で試作したモータ特性表自動生成ツールが対象とするのはブラシ付きDCモータである。しかし、ブラシレスモータやACモータなどには対応していない。
    そのため、それらを用いた回路のシミュレーション結果からモータ特性表を作成できない。
    この問題点は、新たに対象とするモータのcsvファイルを解析し、特性表の要素を算出できるようにすることで、解決できると考える。

\item 特性表の要素のカスタマイズ機能の実装\\
      本論文で試作したモータ特性表自動生成ツールが生成する特性表は、12個の要素を持つ。
      しかし、ユーザがこの要素を変更することはできない。
      この問題点は、ユーザが特性表の要素を設定できるようにすることで、解決できると考える。

% \item 特性グラフを出力形式をユーザが指定できない\\
%       本論文で試作したモータ特性表自動生成ツールは、4つの特性グラフを個別に出力する。
%       しかし、特性グラフは1つのグラフに複数の要素を表示することで、グラフの比較を行いやすくなる。


\end{itemize}



