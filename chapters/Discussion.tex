\chapter{考察}\label{cha:Discussion}
本論文では、モータ特性表自動生成ツールを試作した。

\section{評価}

\subsection{評価方法}
\todo{下記のコマンドを書き換える!}
\newcommand{\mExist}{既存の手法}
\newcommand{\mExtend}{本研究の手法}
\newcommand{\mInput}{仕様書}
\newcommand{\mOutput}{仕様書}

\mExist{}と、\mExtend{}で、作成(生成)に要した時間の比較検証を行った。
その結果を、表\ref{tab:time}に示す。

対象とした\mInput{}は、\ref{cha:domain}節で用いた コード\ref{fig:vdm_park}である。
\mOutput{}を作成する時間を計測した。
生成する\mOutput{}としては、以下を基準とした。\todo{なにをもって完成かを書く!}
\begin{enumerate}
  \item onポイント、offポイント、inポイント、outポイントを出力(記述)する
  \item onポイント、offポイント、outポイントには、着目条件式も出力(記述)する
  \item offポイントには、着目変数も出力(記述)する
  \item 各ポイントには、期待出力と正常系であるかどうかも出力(記述)する
\end{enumerate}

検証に参加したメンバーは本研究室の大学院生\todo{X}人と学部4年生\todo{Y}人であり、
普段からソースコードの読み書きを行い、基本的なプログラミングの知識を有している。
\mInput{}の知識を持たない者も含まれるが、
今回の検証に必要な文法は、事前に他の\mInput{}の例を用いてレクチャーした。
また、\mOutput{}生成についても、事前に他の\mInput{}と\mOutput{}の例を用いてレクチャーした。

人手による検証では、
コード\ref{fig:vdm_park}を印刷した紙を渡し、
\mInput{}を確認後、
\mOutput{}を書き始めてから、\mOutput{}を記述し終えるのに要した時間を計測した。
\todo{なんか}が不正確な場合、間違いを指摘し、
被験者が正しい\mOutput{}を記述した時点で時間計測終了とした。
また、制限時間を\todo{Z}分とし、制限時間を超えた場合、その場で時間計測終了とした。

\mExtend{}による検証では、
\todo{計測はじめから終わりの条件と、使ったPCの仕様を書く}
コマンドライン上での命令操作で、\mExtend{}による\mOutput{}生成を行うのに要した時間を計測した。
また、実験に用いたコンピュータは、OS:macOS 10.14.5、CPU:2.3GHz Intel Core i5、メモリ:16GBである。

\todo{純粋な、実行時間を書く}
なお、JavaのSystem.nanoTime\cite{nanotime}メソッドを用いて、
命令操作を省いた純粋な\mOutput{}生成処理に\mExtend{}が要した時間を計測した結果、
\todo{A}秒であった。

人手による作成と比較した結果、平均で\todo{B}分程の時間短縮を確認できた。
対象にした\mInput{}には、\todo{なにか}独特の文法等は含まれないため、
\todo{なにか}に対する慣れなどの影響は無視できるものと思われる。
また、人手による\mOutput{}生成の場合、ヒューマンエラーも見られた。
\todo{ヒューマンエラーがあったら、具体例を書く}
具体的には、offポイントの記述時に、条件式の解釈を間違え、誤った期待出力を記述してしまった。(例:入力(17、 20)の期待出力を``遊園地チケットは割引価格とならない。(妻の年齢 $<$ 16)''と記述した。)
\mInput{}の規模が拡大すると、人手とコンピュータとの処理効率の差に加えて、
ヒューマンエラーの有無などにより、\mOutput{}生成に要する時間の差は更に拡大していくと思われる。
以上から、\mExtend{}は有用性が向上したと考える。

\begin{table}[tp]
\centering
\caption{コード\ref{fig:vdm_park}の\mOutput{}作成に要した時間の比較}
\label{tab:time}
\begin{tabular}{cc}
\begin{minipage}[c]{0.5\hsize}
  \centering
  \begin{tabular}{c|c}
    被験者  & 時間              \\
    \hline
    \hline
    被験者A & 8m 16s            \\ \hline
    被験者B & 10m 23s           \\ \hline
    被験者C & 30m(制限時間超過) \\ \hline
    被験者D & 24m 04s
  \end{tabular}
\end{minipage} &
\begin{minipage}[c]{0.5\hsize}
  \centering
  \begin{tabular}{c|c}
                 & 時間    \\
    \hline
    \hline
    被験者(平均) & 18m 10s \\ \hline
    BWDM         & 0m 15s
  \end{tabular}
\end{minipage}
\end {tabular}
\end{table}



\subsection{結果}
本論文で試作したモータ特性表自動生成ツールは、

\section{関連研究}

	関連研究について述べる。

\section{ツールの問題点}

以下に、今回作成したモータ特性表自動生成ツールの問題点を示す。

\begin{itemize}
	\item 対応するモータのモデルは1種類しかない\\
		  モータは~種類に分けることができ、今回は1つにしか対応していない。
		  対応できる数を増やす必要がある。
		
\end{itemize}







