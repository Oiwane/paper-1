\chapter{研究の準備}\label{cha:Preparation}
本章では、本研究で必要となる前提知識を説明する。

% \section{モータ作成}\label{motor}

% \subsection{仕様書}\label{siyo}
% \subsection{シミュレータの役割}\label{simu}

\section{モータ特性表}\label{toku}
モータを選ぶ際に、参考にする資料である。\\
今回作成する特性表の各要素の必要性はここで書く?



\section{OpenModelica}\label{OM}
Modelica言語に対応したOSS
    \subsection{modelica}\label{modelica}
微分代数方程式を用いた複合領域の物理システムモデリングのために開発されたオブジェクト指向言語である。\\
        \subsubsection{Modelica標準ライブラリ(MSL)}\label{MSL}
        Modelica言語による様々な物理領域のモデルライブラリを開発しており、
        数学、機械、電気、熱、流体、制御系、状態遷移機械などを含んだフリーのライブラリがリリースされている。

    \subsection{出力}\label{output}
OpenModelicaでは、シミュレーション結果を以下の3つの形式から選択することができる。

\begin{itemize}
    \item matファイル
    \item pltファイル
    \item csvファイル
\end{itemize}

また、OpenModelicaでは、シミュレーション結果をグラフで確認することが可能である。